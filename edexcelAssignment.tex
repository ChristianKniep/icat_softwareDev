\documentclass[11pt]{article}
\usepackage{geometry}
\geometry{hmargin=1cm,top=1cm,bottom=1cm}
\usepackage{tabularx}
\usepackage{wasysym} % checkbox sybols
\usepackage{graphics}
\usepackage{color}
\usepackage{url}

\definecolor{lGray}{gray}{.90}
\newcommand{\code}[1]{\colorbox{lGray}{\texttt{#1}}}

\author{Christian Kniep}
\title{Title}
\date{\today}

\newcommand{\header}[1]{\textbf{#1}}
%----------------------------
\begin{document}
% very ugly table
\begin{tabular*}{170mm}{|p{85mm}|p{76.5mm}|}
\hline
Program Name & Software Development Project\\ \hline
Name ot the Student & Hansen \\ \hline
Roll Number & \\ \hline
Unit No \& Title & \\ \hline
Assignment No. & \\ \hline
Date Due & \\ \hline
Date Submitted & \\ \hline
\end{tabular*}

{\center \section*{Unit Outcomes}}
\tiny
\begin{tabularx}{\textwidth}{| X | X | X | m{13mm} | m{15mm} |}
\hline
\header{Header 1} & \header{Evidence for the criteria} & \header{Feedback} & \header{Assesor's decision} & \header{Internal Verification}\\ \hline
Develop a plan for a project to an agreed specification & design a plan for an agreed project & & & \Square \\ \hline
 & produce a project specification including complete user requirements and design specifications & & & \Square \\ \hline
 & select and use appropriate software planning tools & & & \Square \\ \hline
Develop a solution for the project & select a suitable programming language & & & \Square \\ \hline
 & produce all algorithms, pseudo code, flowchart, data dictionary, programme coding, classes, methods as required & & & \Square \\ \hline
 & review, select and use appropriate testing techniques to validate the project & & & \Square \\ \hline
 & use appropriate software tools to develop project coding & & & \Square \\ \hline
Implement the solution to the system & identify and implement system requirements & & & \Square \\ \hline
 & apply verification and testing required at all levels of the system & & & \Square \\ \hline
 & produce a clear and structured implementation plan & & & \Square \\ \hline
Present and evaluate the project & present your solution in a structured and well organised format & & & \Square \\ \hline
 & produce documentation for all stages of a project and a full report & & & \Square \\ \hline
 & assess the quality of the product compared to the clients� original requirements & & & \Square \\ \hline
\hline
\header{Pass grades awarded} & & & & \\ \hline
\header{Merit grades awarded} & & & & \\ \hline
\header{Distinction grades awarded} & & & & \\ \hline
\header{Professor�s additional feedback and comments} \\ \hline
 \\ \hline
\header{assessor's name} & & \header{Date} \\ \hline
I hereby confirm that this assignment is my own work and not copied from any source.
I have reference the sources from which from where information is obtained by me for this assignment & \header{Students signiture} \\ \hline
\end{tabularx}
\normalsize

{\center \section*{NOTES TO STUDENTS}}
\begin{itemize}
    \item Check carefully the submission date and the instructions given the assignment. Late assignments will not be accepted.
    \item Ensure that you give yourself enough time to complete the assignment by the due date.
    \item Do not leave things such as printing to the last minute � excuse of this nature will not be accepted for failure to hand in the work on time.
    \item You must take responsibility for managing your own time effectively.
    \item If you are unable to hand in your assignment on time and have valid reason such as illness, you may apply (in writing) for extension.
    \item Failure to achieve PASS grade will results in REFERRAL grade being given
    \item Take great care that if you use other people�s work or ideas in your assignment, you probably reference them in your report.
    \item \textbf{NOTE : if you are caught cheated, you could have your grade reduced to zero, or at worst, you could be excluded from the course.}
\end{itemize}
{\center \section*{GRADING}}
\subsection*{Pass}
\begin{enumerate}
    \item All criteria identified in the assignment are met. 
    \item The deadline to submit the design and final report have been met.
\end{enumerate}
\subsection*{Merit}
\begin{enumerate}
    \item All criteria are meet with a good background research
    \item Your soloutions are well planed and are aware of security and maintenance
\end{enumerate}

\subsection*{Distinction}
\begin{enumerate}
    \item You implemented at least two different kinds of resources
    \item The documentation and the presentation could be used by an untrained user
\end{enumerate}

{\center \section*{TITLE}}
{\center Resource Information System}

{\center \section*{SCENARIO}}
A client of yours likes to put in place a service which provided an unspecified user
the capability of getting Information about resources he usualy can not interact with. \\
It should be a simple commandline based tool. The software should be driven by the following principles:
\begin{enumerate}
    \item \textbf{Usability} It should be easy to use even for an user who has no further IT-experience
    \item \textbf{Reliability} It should be a stable, easy to maintain and reliable system
    \item \textbf{Expendability} New features should be easy to implement in the software 
\end{enumerate}
To gather information there should be a dedicated software-piece to place it
within the resource-area where it is supposed to gather information. \\
The software should include at least five resources in the monitoring system.
Since our employees are not that big in computers you may offer us some possible
resource information you have in mind. Maybe including these areas of interest:
\begin{itemize}
    \item \textbf{Filesystem} Maybe free space, occupied space, swap used, etc.
    \item \textbf{Network} Internet available, ping time, ...
    \item \textbf{OS} What OS is running, some configurations
\end{itemize}

{\center \section*{TASK}}
\subsection*{Task 1}
Evaluate the following topics and decide founded on the research what your software will look like
    \paragraph{Client/Server} What are they suppose to be? How much 'inteligence' should either of them have. What are the advantages and disadvantages of rich or small part?
    \paragraph{Resources} Pic at least three resources you want to monitor with this system.
    \paragraph{Protocol} How should the two parts communicate? Define a the communication for the choosen resources.
    \paragraph{Softare Planning Tool} Evaluate an appropiate Tool to keep track on your progress.
\subsection*{Task 2}
Develop a soulution for the Project
\begin{itemize}
    \item Evaluate at least three programing languages for the specifica this project provides
    \item Create pseudocode and class-diagram  for the choosen resources.
    \item Review what you have achieved so far in terms of maintaiance and security.
\end{itemize}
\subsection*{Task 3}
Implement the project in the choosen language
\begin{itemize}
    \item Create a plan which includes a timescale of the implementation states.
    \item Use the Pseudocode and the class-diagram to create the server and the client in the choosen language.
\end{itemize}
\subsection*{Task 4}
documentation, presentation and review
\begin{itemize}
    \item Create a documentation which contains all aspects of the users interaction (how it should be used).
    \item Present your solution in a structured and well organised format
    \item Have a critical view on your software. Try to work out the advantages and the disadvantages. Document known bugs. What would you change in 
\end{itemize}   

{\center \section*{GUIDANCE}}
\subsection*{Languages}
Consider that you are supposed to handle strings and to write a small, flexible piece of software.
\subsection*{Software Development Tools}
Have a look at some mindmap-, gantgraph-tools and tools in which you can plan your class-modell.
\subsection*{Protocols}
Consider the header/body-patern for your protocol and keep it simple.

{\center \section*{REFERENCES}}
\subsection*{Languages}
Have a look at the following programming languages.  
\begin{itemize}
    \item[bash] \url{http://en.wikipedia.org/wiki/Bash_%28Unix_shell%29}
    \item[java] \url{http://en.wikipedia.org/wiki/Java_%28programming_language%29}
    \item[python] \url{http://en.wikipedia.org/wiki/Python_%28programming_language%29}
    \item[c/c++] \url{http://en.wikipedia.org/wiki/C_%28programming_language%29}
\end{itemize}
\subsection*{Software Development Tools}

\subsection*{Protocols}
\url{http://en.wikipedia.org/wiki/Head-Body_Pattern}

\subsection*{Assignments}
\subsubsection*{Unit 1 - 03.09.2010}
You have done good work on the Client-Server issue but you should not forget to have a look at the development tools.\\
The interface is as simple as it should be so it will be easy for the users to interact with the system.
\subsubsection*{Unit 2 - 15.09.2010}
The comparision between the programming languages are quite detailed and well researched. You should redo the class-diagram in the UML formating as we talked about.\\
The pseudocode only shows what the clients output will be. You should produce the pseudocode for the deamon and the interaction as well. The client sends the request to the deamon, the deamon evaluates the request, send the reply to the client and the client shows the output.
Multiple procedures you just have to do ones.
Don't foret to review the outcomes you have produced so far as described in the Task2.
\subsubsection*{Unit 3 - 22.09.2010}
You have done the planning well, so that the implementation was smooth and easy. You have thought about outsource the communication to have a central, compact unit which handles
layer of TCP/IP-communication.
You could have created a library to handle the system-interaction more objectoriented, but for this project its abstract enough.
Further on you have make good comments in your sourcecode. You and others will be able to catch up to this project in the future due to that.
The comments in the server are a little bit weak, but you have proven in the client that you are capable of understanding what you have produced.
\subsubsection*{Unit 4 - 28.09.2010}
The documentation for the user covers all necessary steps. So the user will be able to handle the software without trouble.
In the 'advantages and disadvantages'-section you talked about the advantages of python and that the performance is quite good.
This is not important in this case, because the user will see no performance-boost. The interaction-performance with an user is
subjective for every user and a human won't care about the speed as long as it is below a couple of 100 milliseconds.
You should redo this part and concentrate more on the 'easy to develop'-advantage.
Also missing are the security and asymetric issues.
\subsubsection*{Review / Presentation - 11.10.2010}
You take the advise to improve the assignment. Your presentation was solid and I was pleased by the answers you gave when I asked you about the project. 

\end{document}
