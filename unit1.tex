\documentclass[11pt]{article}
\usepackage{url}
\usepackage[ngerman]{babel}
\usepackage[utf8]{inputenc}
\usepackage{graphics}
\usepackage{color}

\definecolor{lGray}{gray}{.90}
\newcommand{\code}[1]{\colorbox{lGray}{\texttt{#1}}}

\author{Christian Kniep}
\title{Unit1 \\ scope statement}
\date{\today}
%----------------------------
\begin{document}
\maketitle
In this unit you are supposed to create a scope statement.\\
The scope statement is like an offer you make to the customer to match his product concept catalogue.
\section{Basics}
\subsection{product concept catalogue}
If a customer are willing to pay for some piece of software he has some thougths in mind,
what he is intended to do with it.\\
At least he should have some thoughts.\\
The document which includes the intention of the customer is called 'product concept catalog'. Its rather an technical manual.
Mostly it contains the highest expectation of the customer. In some cases it containts things that are not succedable, so you have to keep that in mind. \\
Its highly recommended to talk to the customer what he realisticly expects and to try to come to the same understanding with the customer to bring the project to a good end.
\subsection{scope statement}
A document which describs the product you are able (willing) to offer to the customer is the scope statement.
It contains the key data of the project you are willing to develop for the customer.
Its also kind of a comercial to advertise your statement. It containts \textbf{NO} writen or even planned code at all. \\
Its like a brief manual with no exlplaination of any detail how it will be implemented. \\
More like: \\
{\colorbox{lGray}{The software will provide a login-mechanism, which connects to a Windows-Domain.} \\
{\colorbox{lGray}{If the login is successfull the user can choose from this options: X Y Z}\\
{\colorbox{lGray}{If he choose option X he will be guided to create a new user within the system.}

\paragraph{Evaluation} The scope statement is also used as a guidline to determin whether your delivered product matches all the goals you and your customer have agreed on.
If its not in the scope statement then you don't have to deliver it. But please keep in mind that you always have to had something to bargain with.
If the customer wants a little change during the way you \textbf{won't} say: 'Sorry sir, thats not in our contract.' \\
If the changes to big and you are not willing to do it on your own cost you have to negotiate with your costumer.

\section{scenario}
This would be the 'product concept catalogue' you get from the customer and to which will be referenced during this course.
\subsection{Resource Monitoring Management}
\paragraph{Intention}
We would like to put in place a service which provided and unspezified user
the capability of getting Information about resources he usualy can not interact with.
It should be easy to use even for an user who has no further IT-experience.
\paragraph{Information gathering}
To gather information we would have a dedicated software-piece to place it within the resource-area where it is supposed to gather information.
\paragraph{Interaction}
The interaction with the information-service should be commandline-based, such as: \code{./infoClient getResourceXY}
\paragraph{Resources}
We want at least five resources in the monitoring system. Since our employees are not that big in computers you may offer us some possible resource information you have in mind.
\end{document}